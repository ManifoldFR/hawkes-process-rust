\documentclass{article}
\usepackage[a4paper]{geometry}
\usepackage{mathtools}
\usepackage{amsfonts}
\usepackage{amssymb}

\usepackage[sfdefault,scaled=.85]{FiraSans}

\usepackage{caption}
\usepackage{graphicx}

\newcommand{\RR}{\mathbb R}
\newcommand{\NN}{\mathbb N}
\newcommand{\ZZ}{\mathbb Z}

\begin{document}
This short document aims to detail the mathematical structure and background of the processes simulated in this library.

\section{Poisson processes}

A $D$-dimensional Poisson process is a stochastic process yielding random events $(s_i)_{i}$ in a subset $A \subseteq \RR^d$.
It is governed by a deterministic event rate $\lambda\colon A\longrightarrow \RR_{+}$. The total number of events $N(B)$ in a subset $B\subset A$ follows a Poisson distribution of parameter $\int_A \lambda(s)\;ds$, so that areas wherein $\lambda$ takes larger values have more events.
We denote $(s_i)_{i}\sim \mathcal{PP}(\lambda)$.

\begin{figure}[h]
  \centering
  \includegraphics[width=0.8\textwidth]{../examples/images/{2d_poisson.variable.circle}.png}
  \caption[]{A Poisson process on the unit circle. 4003 events were simulated, the intensity function is $\lambda(r,\theta) = 4000\cos(4\pi(r + 0.06\cos(30\theta)))$.}
\end{figure}

Extension to vector-valued processes is evident, but not discussed here nor implemented in the library.

\section{Hawkes processes}

\textit{Hawkes processes} are a set of time-dependent point processes where the event rate depends on past events. A multivariate Hawkes process has $D$ point process components which are Poisson process conditionally to the process history $\mathcal H_t = \{(s_n, c_n)\}$, with rate
\begin{equation}
  \lambda_k(t|\mathcal H_t) = \lambda_{0,k}(t) + \sum_{j=1}^D \sum_{s \in \mathcal H_t} g_{j,k}(t-s)
\end{equation}
where $g_{j,k}\colon\RR_+\longrightarrow \RR_+$. Each event is marked with an index $c_n \in \{1,\ldots,D\}$ indicating the process on which it occured.

\begin{figure}[h]
  \centering
  \includegraphics[width=0.9\textwidth]{../examples/images/{hawkes_exp.fixed}.png}
  \caption[]{The intensity process $\lambda(t)$ of a one-dimensional Hawkes process.}
\end{figure}



\end{document}